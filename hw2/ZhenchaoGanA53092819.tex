\documentclass{article} % For LaTeX2e
\usepackage{nips15submit_e,times}
\usepackage{hyperref}
\usepackage{url}
%\documentstyle[nips14submit_09,times,art10]{article} % For LaTeX 2.09

\usepackage{amsmath}
\usepackage{graphicx}
\usepackage{listings}             % Include the listings-package
\lstset{language=Python}          % Set your language (you can change the language for each code-block optionally)

\title{CSE253 Homework 2}


\author{
David S.~Hippocampus\\
Department of Computer Science\\
Cranberry-Lemon University\\
Pittsburgh, PA 15213 \\
\texttt{hippo@cs.cranberry-lemon.edu}
}

% The \author macro works with any number of authors. There are two commands
% used to separate the names and addresses of multiple authors: \And and \AND.
%
% Using \And between authors leaves it to \LaTeX{} to determine where to break
% the lines. Using \AND forces a linebreak at that point. So, if \LaTeX{}
% puts 3 of 4 authors names on the first line, and the last on the second
% line, try using \AND instead of \And before the third author name.

\newcommand{\fix}{\marginpar{FIX}}
\newcommand{\new}{\marginpar{NEW}}

%\nipsfinalcopy % Uncomment for camera-ready version

\begin{document}


\maketitle


\section{Problems I}


By the independence assumption on the $\epsilon^{i}$, we have the following likelihood function:



\begin{equation}
    \begin{array}{rcl}
       L(\theta) & = &  \prod\limits_{i = 1}^{m} p(y^{i} | x^{i} ; \theta)  \\
       & = &  \prod\limits_{i = 1}^{m} \frac{1}{\sqrt{2\pi} \sigma} exp(- \frac{(y^{(i)}   -   \theta^Tx^{(i)}  ) ^ 2}{2\sigma^2})
    \end{array}
\end{equation}

we should choose θ to maximize $L(\theta)$, it is same to maximize following function :

\begin{equation}
    \begin{array}{rcl}
       \ell(\theta) & = & \log{L(\theta)} \\
       & = &  \log {(\prod\limits_{i = 1}^{m} \frac{1}{\sqrt{2\pi} \sigma} exp(- \frac{(y^{(i)}   -   \theta^Tx^{(i)}  ) ^ 2}{2\sigma^2}))} \\
       & = & \sum\limits_{i = 1}^{m} \log {(\frac{1}{\sqrt{2\pi} \sigma} exp(- \frac{(y^{(i)}   -   \theta^Tx^{(i)}  ) ^ 2}{2\sigma^2}))} \\
       & = & m \log{ \frac{1}{\sqrt{2\pi} \sigma}  } - \frac{1}{\sigma^2} \cdot \frac{1}{2} \sum\limits_{i=1}^{m}(   y^{(i)}   -   \theta^Tx^{(i)}   )^2
    \end{array}
\end{equation}

Hence, maximizing $\ell(\theta)$ gives the same answer as minimizing \\

\begin{equation}
    \begin{array}{rcl}
       \frac{1}{2} \sum\limits_{i=1}^{m}(   y^{(i)}   -   \theta^Tx^{(i)}   )^2
    \end{array}
\end{equation}

So, we can conclude that finding the optimal parameter $\theta$ for the above linear regression problem on the dataset $D = \{  (x^{1},t^{1}), ... , (x^{(N)},t^{(N)})  \}$ is equal to finding the $\theta$ that minimizes the SSE:
\begin{equation}
    \begin{array}{rcl}
       \theta^* & = & argmin_\theta \sum\limits_{i=1}^{N}(  t^i - f((1,x^{(i)}), \theta) )^2
    \end{array}
\end{equation}



%% \subsection{Double-blind reviewing}

%% This year we are doing double-blind reviewing: the reviewers will not know 
%% who the authors of the paper are. For submission, the NIPS style file will 
%% automatically anonymize the author list at the beginning of the paper.

%% Please write your paper in such a way to preserve anonymity. Refer to
%% previous work by the author(s) in the third person, rather than first
%% person. Do not provide Web links to supporting material at an identifiable
%% web site.

%%\subsection{Electronic submission}
%%
%% \textbf{THE SUBMISSION DEADLINE IS June 5, 2015. SUBMISSIONS MUST BE LOGGED BY
%% 23:00, June 5, 2015, UNIVERSAL TIME}

%% You must enter your submission in the electronic submission form available at
%% the NIPS website listed above. You will be asked to enter paper title, name of
%% all authors, keyword(s), and data about the contact
%% author (name, full address, telephone, fax, and email). You will need to
%% upload an electronic (postscript or pdf) version of your paper.

%% You can upload more than one version of your paper, until the
%% submission deadline. We strongly recommended uploading your paper in
%% advance of the deadline, so you can avoid last-minute server congestion.
%%
%% Note that your submission is only valid if you get an e-mail
%% confirmation from the server. If you do not get such an e-mail, please
%% try uploading again. 

\section{Problems II}

(a) \\

\begin{equation}
    \begin{array}{rcl}
      	 \delta_k & = & - \frac{\partial{E}}{\partial{a_k}} \\
	 & = & - \frac{\partial{E}}{\partial{\tilde{g}(a_k)}} \frac{\partial{\tilde{g}(a_k)}} {\partial{a_k}} \\
	 & = & \frac{(y_k - t_k)}{\tilde{g}'(a_k)} \\
	 & = & (t_k - y_k)
    \end{array}
\end{equation}


\begin{equation}
    \begin{array}{rcl}
      	 \delta_j & = & - \frac{\partial{E}}{\partial{a_j}} \\
	 & = & - \sum\limits_{k=1}^{c} \frac{\partial{E}}{\partial{a_k}} \frac{\partial{a_k}}{\partial{a_j}} \\
	 & = & - \sum\limits_{k=1}^{c} \frac{\partial{E}}{\partial{a_k}} \frac{\partial{a_k}}{\partial{z_j}} \frac{\partial{z_j}}{\partial{a_j}} \\
	 & = & g'(a_j) \sum\limits_{k=1}^{c}  \delta_k w_{jk}
    \end{array}
\end{equation}

(b) \\

\begin{equation}
    \begin{array}{rcl}
      	 w_{jk}(t+1) & = & w_{jk}(t) - \alpha \frac{\partial{E}}{\partial{w_{jk}(t)}} \\
	 & = & w_{jk}(t) - \alpha \frac{\partial{E}}{\partial{a_k}}z_j \\
	 & = & w_{jk}(t) + \alpha \delta_k z_j
    \end{array}
\end{equation}

\begin{equation}
    \begin{array}{rcl}
      	 w_{ij}(t+1) & = & w_{ij}(t) - \alpha \frac{\partial{E}}{\partial{w_{ij}}} \\
	 & = & w_{ij}(t) - \alpha \frac{\partial{E}}{\partial{a_j}}   \frac{\partial{a_j}}{\partial{w_{ij}}} \\
	 & = & w_{ij}(t) + \alpha  \delta_j x_i
    \end{array}
\end{equation}

(c) \\
T is the target, Y is output. Assuming $Z : m \times n$

\begin{equation}
    \begin{array}{rcl}
      	 w_{HO} & = & w_{HO} + \alpha * ( (T - Y)^T \cdot Z) \\
	 w_{IH} & = & w_{IH} + \alpha * (  Z * ( (1)_{m \times n} - Z)   * (   (T - Y) \cdot W_{HO}^T )     ) \cdot X
    \end{array}
\end{equation}

* is element-wise multiplication, $\cdot$ is matrix and vector multiplication

(d) \\ i.
I use scipy to zscore raw data. \\

ii. The gradient using numerical differences with the one using backpropagation is very close. So my algorithm produce right gradient.

iii. First, I use Normal distribution with scale 0.1 to initialize parameters. Then I divide my data into 30 racks and run training on these racks individually instead of the total data. I also test different iteration numbers and different learning rate.

\begin{figure}[htbp] %  figure placement: here, top, bottom, or page
   \centering
   \includegraphics[width=5in]{img/p4.png} 
\end{figure}

\pagebreak

(e) \\
\begin{figure}[htbp] %  figure placement: here, top, bottom, or page
   \centering
   \includegraphics[width=5in]{img/p5.png} 
\end{figure}

When $\lambda = 0.001$, we can see that the accuracy is about 92\%,worse than that without regularization. When $\lambda = 0.0001$, the accuracy is almost same as that without regularization. I guess it is because when $\lambda$ is too big, then the penalty will account for a lot and influence the original training procedure. Because without regularisation, my accuracy is above 96\%, it is hard to improve.

(f) \\
\begin{figure}[htbp] %  figure placement: here, top, bottom, or page
   \centering
   \includegraphics[width=5in]{img/p6.png} 
\end{figure}

With momentum, the final accuracy is almost same as before. But the convergence speed is much faster. We can find that momentum simply adds a fraction m of the previous weight update to the current one. As a result, when the gradient keeps pointing in the same direction, this will increase the size of the steps taken towards the minimum.

(g) \\

As for Tanh : the update rule is almost the same, the only difference is $g'(a_j)  = 1 - Tanh(a_j)^2$  \\

As for ReLU : 

\begin{equation}
    ReLU(x)'=
   \begin{cases}
   1 &\mbox{if $x > 0$}\\
   0 &\mbox{otherwise}
   \end{cases}
  \end{equation}
  
  So we set $g'(a_j)  = ReLU(a_j)'$ \\

\begin{figure}[htbp] %  figure placement: here, top, bottom, or page
   \centering
   \includegraphics[width=5in]{img/p71.png} 
\end{figure}

\begin{figure}[htbp] %  figure placement: here, top, bottom, or page
   \centering
   \includegraphics[width=5in]{img/p72.png} 
\end{figure}

When $f(z) = Tanh$, test set accuracy is above 95\%. When $f(z) = ReLU(z)$, test set accuracy is about 97.12\%  and train set accuracy is 100\%, so amazing. ReLU is a very useful activation function.

Moreover, ReLU is easier to calculate than Tanh and logistic function. So When I use ReLU, the training procedure is much faster.

\pagebreak

(h) \\

i. It seems that when using more nodes, we can converge faster in each iteration and get more accurate result. Of course, The calculation takes more CPU time. Below is the result of 25,50 and 100 hidden nodes.

\begin{figure}[htbp] %  figure placement: here, top, bottom, or page
   \centering
   \includegraphics[width=5in]{img/p81.png} 
\end{figure}

ii. 

\begin{figure}[htbp] %  figure placement: here, top, bottom, or page
   \centering
   \includegraphics[width=5in]{img/p82.png} 
\end{figure}

Two hidden layers runs slower than one hidden layers.  But they produce close results on test data.

\begin{lstlisting}[frame=single]  % Start your code-block


\end{lstlisting}

\end{document}